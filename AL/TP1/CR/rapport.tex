\documentclass{article}
%packages
\usepackage{graphicx}
\usepackage[utf8]{inputenc}
\usepackage[T1]{fontenc}
\usepackage[frenchb]{babel}
\usepackage[a4paper]{geometry}
\usepackage{minted}
%\usepackage{hyperref}
\usepackage{textcomp}

\begin{document}
%title
\begin{titlepage}
	\vspace{-20px}
	\begin{tabular}{l}
		\textsc{Blin} Sébastien\\
		\textsc{Collin} Pierre-Henri
	\end{tabular}
	\hfill \vspace{10px}\includegraphics[scale=0.1]{../../../esir.png}\\
	\vfill
	\begin{center}
		\Huge{École supérieure d'ingénieurs de Rennes}\\
		\vspace{1cm}
		\LARGE{2ème Année}\\
		\large{Parcours Informatique}\\
		\vspace{0.5cm}\hrule\vspace{0.5cm}
		\LARGE{\textbf{Architecture Logicielle}}\\
		\Large{Compte-Rendu TP n\textdegree 1}
		\vspace{0.5cm}\hrule
		\vfill
		\vfill
	\end{center}
	\begin{flushleft}
		\Large{Sous l'encadrement de~:}\\
		\vspace{0.2cm}
		\large{\textsc{Temple} Paul}
	\end{flushleft}
	\vfill
\end{titlepage}

\section{Descriptif du projet}
Ce premier TP d'Architecture Logicielle nous a permis de mettre en place tout un processus de compilation de la gestion de dépendance et de l'installation du projet en local en passant  par l'automatisation des tests avec \emph{Maven}, à la relecture du code (QA) par un tiers, analyse statique avec \emph{Sonar} jusqu'à l'automatisation des builds sur différentes plateformes avec \emph{Jenkins}
\section{Maven}
\subsection{Description}
Maven est un outil de gestion et de compréhension de projets. Maven fournit différentes fonctionnalités de :
Construction, documentation, rapport, gestion des dépendances, gestion des sources, mise à jour de projet et déploiement. \\\\
Chaque projet Maven est décrit par un fichier \emph{pom.xml} présent à la racine du répertoire du projet. Ce fichier
contient tous les éléments permettant de gérer le cycle de vie du projet.
\subsection{Utilisation}
\subsubsection{Génération de projets}
Pour créer un simple projet Java, on utilise l'archetype (un template de projet qui permet de gagner du temps et de respecter une certaine convention) Maven suivant : \\
\textit{maven-archetype-quickstart}. \\
La commande utilisée est la suivante : \textit{mvn archetype:generate} \\
On doit fournir un \textit{groupId} (identifiant du groupe de projets) et un \textit{artefactid} (identifiant du projet).\\
\subsubsection{Génération de rapports}
Pour générer la javadoc, on utilise le plugin \textit{maven-javadoc-plugin} qu'on ajoute dans la section <reporting> du pom.xml :
\begin{verbatim}
<plugin>
        <groupId>org.apache.maven.plugins</groupId>
        <artifactId>maven-javadoc-plugin</artifactId>
        <version>2.9.1</version>
</plugin>
\end{verbatim}
Ensuite, nous avons lancé la commande \textit{mvn site} qui génére un site web pour le projet et qui contient la javadoc du projet. \\\\
Nous avons également ajouté le plugin \textit{maven-checkstyle-plugin} qui est un outil de contrôle du code et qui permet de vérifier le style du code source. La norme de codage utilisée par défaut est celle de Sun Microsystems (\textit{sun\_checks.xml}). Parmi les règles prédéfinies, on trouve aussi celles de Google (\textit{google\_checks.xml}). Par ailleurs, il est tout à fait possible d'utiliser un ensemble de règles personnalisées. Une nouvelle section apparait alors sur le site. \\\\
Enfin, nous avons ajouté le plugin \textit{maven-jxr-plugin} qui permet de référencer et de trouver plus facilement des lignes de code spécifiques.	Ici, l'association avec le plugin checkstyle permet de lier automatiquement le numéro de la ligne pour chaque règle non respectée et le code source.
\subsubsection{Génération de packages}
Afin de personnaliser l'étape de compilation et l'étape de création des artefacts, nous avons ajouté le plugin \textit{maven-compiler-plugin} dans la section <build> du pom.xml qui permet de spécifier à Maven les options de configuration pour l'étape de compilation. En effet, la section <reporting> est dédiée à la génération des rapports. \\\\
\begin{verbatim}
<plugin>
        <groupId>org.apache.maven.plugins</groupId>
        <artifactId>maven-compiler-plugin</artifactId>
        <version>3.1</version>
        <configuration>
          <source>1.7</source>
          <target>1.7</target>
        </configuration>
</plugin>
\end{verbatim}
Nous avons également ajouté le plugin \textit{maven-jar-plugin} qui permet à Maven de packager le projet.\\
\begin{verbatim}
  <plugin>
        <groupId>org.apache.maven.plugins</groupId>
        <artifactId>maven-jar-plugin</artifactId>
        <version>2.6</version>
</plugin>
\end{verbatim}
Le dernier plugin ajouté est \textit{maven-assembly-plugin} qui permet de créer des archives.
\begin{verbatim}
<plugin>
        <artifactId>maven-assembly-plugin</artifactId>
        <version>2.5.5</version>
        <configuration>
          <descriptors>
            <descriptor>src/assembly/assemblySrc.xml</descriptor>
            <descriptor>src/assembly/assemblyBin.xml</descriptor>
          </descriptors>
          <finalName>TP1-Maven-${project.version}</finalName>
        </configuration>
        <executions>
          <execution>
            <id>make-assembly</id>
            <phase>package</phase>
            <goals>
              <goal>single</goal>
            </goals>
          </execution>
        </executions>
</plugin>
\end{verbatim}
\subsubsection{Tests}
Pour éxécuter les tests du projet à chaque goal test ou à chaque compilation , on utilise le plugin
\textit{maven-surefire-plugin}.
\subsection{Pourquoi l'utiliser ?}
Maven sert à compiler le code et s'occuper de la phase de production. Il s'occupe pour nous de la gestion des dépendances. On a juste à décrir les dépendances du projet. C'est très pratique car on ne stocke pas toujours les librairies tierses sur les serveurs de collaboration, simplement une référence. Si un collègue récupère le projet et qu'il n'a pas les librairies, il ne peut pas le compiler, d'où Maven. \\\\
Pour les gros projets séparés en plusieurs modules, Maven permet d'avoir une hiérarchie de projet, de la déclarer et d'avoir des choses en commun. Il permet de fixer les versions des librairies utilisées.

\section{SonarQube}
\subsection{Description}
SonarQube est une plateforme open-source utilisée en QA. Cet outil sert à vérifier (analyse statique) les règles de mise en forme, la complexité d'un programme, les commentaires, les bugs potentiels. De nombreux modules existent pour ajouter la gestion d'autres languages. Les métriques utilisées sont :
\begin{itemize}
	\item Le nombre de lignes de code (ainsi que le nombre de fichiers, répertoires, de fonctions, de classes, etc)
	\item Le pourcentage de code dupliqué
	\item La complexité (mesurée en nombres de fonctions/classes/fichiers)
	\item Les problèmes de code (des issues sont ouvertes, classées par ordre d'importance (Bloquant, critique, majeur, mineur, informatif))
\end{itemize}
Les problèmes détectés par l'outil sont paramétrables à l'aide d'un jeu de règles. Par défaut pour notre projet, nous avons utilisé le profil Sonar Way (Java) qui contient 192 règles par défaut. Ces règles sont activables/désactivables. De plus des règles peuvent être créées manuellement (nous n'avons pas testé cette fonctionnalité). Généralement les règles contiennent une description précise.
\subsection{Utilisation}
L'utilisation de cet outil fut assez basique. Elle consistait à lancer \emph{Sonar} et à éxécuter~:
\begin{verbatim}
mvn sonar:sonar
\end{verbatim}
Ce qui permettait d'obtenir le projet sur l'interface de \emph{Sonar}. Ensuite, nous avons pu parcourir l'interface et paramétrer les profils pour détecter plus d'erreurs ou enlever les erreurs reconnues.
\subsection{Pourquoi l'utiliser ?}
Les outils d'analyse statiques sont très utilisés dans les projets, car un développeur ne travaille généralement pas seul. Du coup ce type d'outil est utile pour les raisons suivantes~:
\begin{itemize}
	\item Un code propre est généralement un code facile à lire, à comprendre : ce qui facilite le travail aux personnes travaillant avec le développeur.
	\item Une erreur dans le code est rapidement arrivée. Une analyse statique permet d'éviter des bugs faciles à trouver comme la double libération d'une zone mémoire, l'utilisation d'un pointeur non initialisé, etc.
\end{itemize}
Il existe d'autres outils d'analyse statiques qui n'ont pas été abordés dans ce TP, notament l'analyseur statique de \emph{clang}.
\begin{figure}
	\begin{center}
		\includegraphics[scale=0.3]{img/sonarqube}
		\caption{Le paramétrage des règles du profil \emph{Sonar Way (Java)}}
		\label{fig:Sonarqube}
	\end{center}
\end{figure}

\section{Jenkins}
\subsection{Description}
\emph{Jenkins} est un serveur d'automatisation de compilation. Cet outil permet d'automatiser des compilations, des tests et le déploiement d'une application. De plus il est personnalisable avec des centaines de plugins.
\begin{figure}
	\begin{center}
		\includegraphics[scale=0.3]{img/jenkins}
		\caption{L'historique des builds de \emph{Jenkins}}
		\label{fig:Jenkins}
	\end{center}
\end{figure}
\subsection{Utilisation}
Ce TP nous a permis d'automatiser les builds d'un git distant. Dans notre cas, il s'agit du dépot disponible à l'adresse \url{https://github.com/AmarOk1412/MinisTPS8}. Ce dépot contenant plusieurs matières, nous avons créé un projet de type Multi-configurations et dans la partie \emph{Build}, nous avons ajouté un appel à un script shell contenant~:
\begin{minted}{bash}
	sh -c 'cd AL/TP1 && mvn clean package install'
\end{minted}
Bien sûr, ce script peut-être complété pour compiler tous les TPs et effectuer tous les tests.\\
Nous avons ensuite paramétré le \emph{pom.xml} de \emph{Maven} afin d'avoir un package propre en sortie de compilation. L'archive générée par la compilation automatisée est disponible dans le workspace (tout comme la sortie de la console, ce qui peut-être utile en cas de bug).
\subsection{Pourquoi l'utiliser ?}
Les outils d'intégration continue sont utiles car ils permettent de~:
\begin{itemize}
	\item Savoir quel commit a apporté un bug (automatisation des tests) et permet de savoir si quelque chose c'est mal déroulé sur une plateforme ce qui permet de prévenir les développeurs (mail) et de régler le problème rapidement.
	\item Pouvoir automatiser le déploiement de l'application sans suivi (et paramétrer des cycles de déploiement (par exemple tous les jours pour une version alpha, toutes les semaines pour une béta et tous les mois pour une version stable)).
\end{itemize}
D'autres outils existent comme \emph{Travis} qui est aussi utilisé sur le dépot (\url{https://travis-ci.org/AmarOk1412/MinisTPS8}).
\begin{figure}
	\begin{center}
		\includegraphics[scale=0.3]{img/travis}
		\caption{L'outil \emph{Travis-CI} sur le dépot du projet}
		\label{fig:Travis}
	\end{center}
\end{figure}
\end{document}
